\documentclass[12pt,a4paper]{article}
\usepackage[utf8]{inputenc}
\usepackage[T1]{fontenc}
\usepackage[slovene]{babel}
\usepackage{lmodern}
\usepackage{enumitem}
\usepackage{amsmath}
\usepackage{amssymb}
\usepackage{float} 
\usepackage{url} 
%\usepackage{hyperref} 
\usepackage{graphicx} 
\usepackage[margin=0.3in]{geometry}
\usepackage{multirow}


\title{Ali si že mojster v potenciranju in korenjenju?}
\author{}
\date{}

\begin{document}

\maketitle

\begin{enumerate}
    \item Zapiši kot produkt in izračunaj vrednost potence, kjer lahko to narediš.
    \begin{enumerate}
        \item $(-3)^3=$ \\
        \item $\left(\frac{1}{4}\right)^5=$ \\
        \item $(-b)^4=$ \\ \\
    \end{enumerate}

    \item Produkt oziroma količnik zapiši kot potenco.
    \begin{enumerate}
        \item $5^7\cdot 5^2=$ \\
        \item $0,25^4\cdot 8^4=$ \\
        \item $\left(-\frac{2}{5}\right)^{15}\cdot\left(-\frac{2}{5}\right)^3:\left(-\frac{2}{5}\right)^{18}=$ \\
        \item $\frac{a^5\cdot a^7}{a^3\cdot a^8}=$ \\
        \item $\left(\left(1,8\right)^2\right)^3=$ \\ \\
    \end{enumerate}
    
    \item Zapiši izraz po besedilu in izračunaj njegovo vrednost.
    \begin{enumerate}
        \item Vsota kvadratov števil $10$ in $13$. \\ \\
        \item Kvadrat produkta ulomkov $\frac{8}{9}$ in $\frac{3}{4}$. \\ \\ \\
    \end{enumerate}

    \item Izpolni preglednico.
    \begin{table}[H]
        \centering
        \addtolength{\tabcolsep}{7pt}
        \renewcommand{\arraystretch}{2}                
        \begin{tabular}{||c||c|c|c|c|c|c||} 
            \hline \hline
                    $x$ & $12$ & & $-0,13$ & & $70$ &  \\ 
            \hline
                    $x^2$ & & $\frac{4}{9}$ & & $0,0144$ & & $1210000$  \\ 
            \hline \hline
        \end{tabular}
    \end{table}

\end{enumerate}

\newpage

\begin{enumerate}[resume]

    \item Čim bolj spretno izračunaj.
    \begin{enumerate}
        \item $\sqrt{25\cdot 36}=$ \\
        \item $\sqrt{\frac{16}{81}\cdot\frac{64}{49}}=$ \\
        \item $\sqrt{\frac{3}{8}:\frac{27}{32}}=$ \\ \\
    \end{enumerate}

    \item Delno koreni.
    \begin{enumerate}
        \item $\sqrt{50}=$ \\
        \item $\sqrt{96}=$ \\ \\
    \end{enumerate}

    \item Racionaliziraj ulomke in jih okrajšaj.
    \begin{enumerate}
        \item $\frac{7}{\sqrt{14}}=$ \\
        \item $\frac{3}{2\sqrt{2}}=$ \\ \\
    \end{enumerate}

    \item Izračunaj vrednosti izrazov.
    \begin{enumerate}
        \item $2\cdot\sqrt{121}+3^2\cdot\left(\sqrt{144}-\sqrt{196}\right)=$ \\ \\ \\ \\ \\ \\
        \item $\frac{\sqrt{3^3-\sqrt{121}}:2^2}{2^4-3\sqrt{25}}=$ \\ \\ \\ \\ \\
    \end{enumerate}

\end{enumerate}

\end{document}