\section{Potence}

\begin{frame}
    \sectionpage
\end{frame}

\begin{frame}
    \tableofcontents[currentsection, hideothersubsections]
\end{frame}

    \subsection{Potence}

        % \begin{frame}
        %     \frametitle{Potence}
        % \end{frame}

    \subsection{Množenje in deljenje potenc z enakimi osnovami}

        % \begin{frame}
        %     \frametitle{Množenje in deljenje potenc z enakimi osnovami}
        % \end{frame}

    \subsection{Potenciranje produkta in količnika}

        % \begin{frame}
        %     \frametitle{Potenciranje produkta in količnika}
        % \end{frame}

    \subsection{Kvadriranje racionalnih števil}

        % \begin{frame}
        %     \frametitle{Kvadriranje racionalnih števil}
        % \end{frame}

    \subsection{Kvadratni koren racionalnih števil}

        % \begin{frame}
        %     \frametitle{Kvadratni koren racionalnih števil}
        % \end{frame}

    \subsection{Izrazi s potencami in koreni}

        % \begin{frame}
        %     \frametitle{Izrazi s potencami in koreni}
        % \end{frame}

    % \subsection{Špela se preizkusi}

    %     \begin{frame}
    %         \frametitle{Špela se preizkusi}
    %     \end{frame}

    \subsection{Ali si že mojster v potenciranju in korenjenju?}

        \begin{frame}[t]
            \frametitle{Ali si že mojster v potenciranju in korenjenju?}

            \begin{alertblock}{1. Zapiši kot produkt in izračunaj vrednost potence, kjer lahko to narediš.}
                \begin{itemize}
                    \item $(-3)^3=$ \\ ~ \\ ~ \\
                    \item $\left(\frac{1}{4}\right)^5=$ \\ ~ \\ ~ \\
                    \item $(-b)^4=$ \\ ~ \\ ~ \\
                \end{itemize}
            \end{alertblock}
        \end{frame}

        \begin{frame}[t]
            \begin{alertblock}{2. Produkt oziroma količnik zapiši kot potenco.}
                \begin{itemize}
                    \item $5^7\cdot 5^2=$ \\ ~ \\ 
                    \item $0,25^4\cdot 8^4=$ \\ ~ \\ 
                    \item $\left(-\frac{2}{5}\right)^{15}\cdot\left(-\frac{2}{5}\right)^3:\left(-\frac{2}{5}\right)^{18}=$ \\ ~ \\ ~ \\
                    \item $\frac{a^5\cdot a^7}{a^3\cdot a^8}=$ \\ ~ \\ ~ \\
                    \item $\left(\left(1,8\right)^2\right)^3=$ \\ ~ \\ 
                \end{itemize}
            \end{alertblock}
        \end{frame}

        \begin{frame}[t]
            \begin{alertblock}{3. Zapiši izraz po besedilu in izračunaj njegovo vrednost.}
                \begin{itemize}
                    \item Vsota kvadratov števil $10$ in $13$. \\ ~ \\ ~ \\
                    \item Kvadrat produkta ulomkov $\frac{8}{9}$ in $\frac{3}{4}$. \\ ~ \\ ~ \\
                \end{itemize}
            \end{alertblock}

            \begin{alertblock}{4. Izpolni preglednico.}
                \begin{table}[H]
                    \centering
                    \addtolength{\tabcolsep}{7pt}
                    \renewcommand{\arraystretch}{2}                
                    \begin{tabular}{||c||c|c|c|c|c|c||} 
                        \hline \hline
                                $x$ & $12$ & & $-0,13$ & & $70$ &  \\ 
                        \hline
                                $x^2$ & & $\frac{4}{9}$ & & $0,0144$ & & $1210000$  \\ 
                        \hline \hline
                    \end{tabular}
                \end{table}
            \end{alertblock}
        \end{frame}

        \begin{frame}[t]
            \begin{alertblock}{5. Čim bolj spretno izračunaj.}
                \begin{itemize}
                    \item $\sqrt{25\cdot 36}=$ \\ ~ \\ ~ \\
                    \item $\sqrt{\frac{16}{81}\cdot\frac{64}{49}}=$ \\ ~ \\ ~ \\
                    \item $\sqrt{\frac{3}{8}:\frac{27}{32}}=$ \\ ~ \\ ~ \\
                \end{itemize}
            \end{alertblock}
        \end{frame}

        \begin{frame}[t]
            \begin{alertblock}{6. Delno koreni.}
                \begin{itemize}
                    \item $\sqrt{50}=$ \\ ~ \\
                    \item $\sqrt{96}=$ \\ ~ \\
                \end{itemize}
            \end{alertblock}

            \begin{alertblock}{7. Racionaliziraj ulomke in jih okrajšaj.}
                \begin{itemize}
                    \item $\frac{7}{\sqrt{14}}=$ \\ ~ \\ ~ \\
                    \item $\frac{3}{2\sqrt{2}}=$ \\ ~ \\ ~ \\
                \end{itemize}
            \end{alertblock}
        \end{frame}

        
        \begin{frame}[t]
            \begin{alertblock}{8. Izračunaj vrednosti izrazov.}
                \begin{itemize}
                    \item $2\cdot\sqrt{121}+3^2\cdot\left(\sqrt{144}-\sqrt{196}\right)=$ \\~ \\ ~ \\ ~ \\ ~ \\ ~ \\
                    \item $\frac{\sqrt{3^3-\sqrt{121}}:2^2}{2^4-3\sqrt{25}}=$ \\ ~ \\ ~ \\ ~ \\ ~ \\ ~ \\
                \end{itemize}
            \end{alertblock}
        \end{frame}