\section{Računanje z ulomki}

\begin{frame}
    \sectionpage
\end{frame}

\begin{frame}
    \tableofcontents[currentsection, hideothersubsections]
\end{frame}

    \subsection{Ulomki z enakimi imenovalci}

        \begin{frame}[t]
            \frametitle{Ulomki z enakimi imenovalci}

            \begin{alertblock}{}
                Ulomke z enakimi imenovalci \textbf{seštevamo} tako, da \textbf{seštejemo števce}, \textbf{imenovalce} pa \textbf{prepišemo}.
                $$ \mathbf{\frac{a}{c}+\frac{b}{c}=\frac{a+b}{c}}, $$ pri pogoju, da $c\neq 0$.

                Ulomke z enakimi imenovalci \textbf{odštevamo} tako, da \textbf{imenovalec prepišemo}, števec pa izračunamo tako, da \textbf{od števca prvega ulomka odštejemo števec drugega ulomka}.
                $$ \mathbf{\frac{a}{c}-\frac{b}{c}=\frac{a-b}{c}}, $$ pri pogoju, da $a\leq b$ in $c\neq 0$.
            \end{alertblock}
        \end{frame}

        \begin{frame}[t]
            \begin{block}{POMNI}
                Rezultat zapišemo s celim delom in delom manjšim od $1$ ter ga okrajšamo.
            \end{block}
        \end{frame}

    \subsection{Seštevanje ulomkov}

        \begin{frame}[t]
            \frametitle{Seštevanje ulomkov}

            \begin{alertblock}{Seštevanje ulomkov z različnimi imenovalci}
                Ulomke z \textbf{različnimi imenovalci seštevamo} tako, da jih najprej \textbf{razširimo na skupni imenovalec}, imenovalec prepišemo, števce pa seštejemo.
            \end{alertblock}

            \begin{block}{POMNI}
                Dobljeni rezultat zapišemo s celim delom in delom, manjšim od $1$.
            \end{block}

            \begin{block}{POMNI}
                Rezultat vedno zapišemo kot okrajšan ulomek.
            \end{block}

        \end{frame}

    \subsection{Odštevanje ulomkov}

        \begin{frame}[t]
            \frametitle{Odštevanje ulomkov}

            \begin{alertblock}{Odštevanje ulomkov z različnimi imenovalci}
                Ulomke z \textbf{različnimi imenovalci odštevamo} tako, da jih najprej \textbf{razširimo na skupni imenovalec}, imenovalec prepišemo, od števca zmanjševanca (prvega ulomka) pa odštejemo števec odštevanca (drugega ulomka). 
            \end{alertblock}

            \begin{block}{POMNI}
                Če moramo zaporedoma odšteti več odštevancev, odštevance seštejemo in nato odštejemo njihovo vsoto.
            \end{block}

        \end{frame}

    \subsection{Množenje ulomka z naravnim številom}

        \begin{frame}[t]
            \frametitle{Množenje ulomka z naravnim številom}

            \begin{alertblock}{}
                Ulomek \textbf{množimo z naravnim številom} tako, da \textbf{števec pomnožimo z naravnim številom}, \textbf{imenovalec} pa \textbf{prepišemo}.
                $$ \mathbf{n\cdot\frac{a}{b}=\frac{n\cdot a}{b}}, $$ pri pogoju, da $b\neq 0$.
            \end{alertblock}        
            
            \begin{exampleblock}{POZOR}
                $$ n\cdot\frac{a}{b}\neq n\frac{a}{b} $$
            \end{exampleblock}
        \end{frame}


    \subsection{Množenje ulomka z ulomkom}

        \begin{frame}[t]
            \frametitle{Množenje ulomka z ulomkom}

            \begin{alertblock}{}
                Ulomek \textbf{množimo} z ulomkom tako, da \textbf{pomnožimo števec s števcem} in \textbf{imenovalec z imenovalcem}.
                $$ \mathbf{\frac{a}{b}\cdot\frac{c}{d}=\frac{a\cdot c}{b\cdot d}}, $$ pri pogoju, da $b\neq 0, d\neq 0$.
            \end{alertblock}

            \begin{exampleblock}{POZOR}
                $$ m\frac{a}{b}\cdot n\frac{c}{d}\neq m\cdot n\frac{a\cdot c}{b\cdot d} $$
            \end{exampleblock}

            \begin{block}{POMNI}
                Rezultat naj bo vedno okrajšani ulomek. Če je mogoče, naj bo zapisan s celim delom  in ulomkom, manjšim od $1$.
            \end{block}

        \end{frame}

    \subsection{Deljenje ulomka z naravnim številom}

        \begin{frame}[t]
            \frametitle{Deljenje ulomka z naravnim številom}

            \begin{alertblock}{}
                \textbf{Ulomek delimo z naravnim številom} na dva načina:
                \begin{enumerate}
                    \item \textbf{števec} ulomka \textbf{delimo} z naravnim številom: $$\mathbf{\frac{a}{b}:n=\frac{a:n}{b}};$$
                    \item \textbf{imenovalec} ulomka \textbf{pomnožimo} z naravnim številom: $$\mathbf{\frac{a}{b}:n=\frac{a}{b\cdot n}}.$$
                \end{enumerate}
            \end{alertblock}                
            
            \begin{exampleblock}{POZOR}
                Drugi način je vedno mogoč, prvi pa le, če je števec ulomka deljiv z danim naravnim številom.
            \end{exampleblock}
        \end{frame}

    \subsection{Deljenje ulomka z ulomkom}

        \begin{frame}[t]
            \frametitle{Deljenje ulomka z ulomkom}

            \begin{alertblock}{Obratni ulomek}
                Obratna ulomka sta ulomka, katerih produkt je enak $1$.
                $$ \mathbf{\frac{a}{b}\cdot\frac{b}{a}=1} $$
            \end{alertblock}

            \begin{alertblock}{Deljenje ulomkov}
                Ulomek delimo z drugim ulomkom tako, da ga pomnožimo z obratno vrednostjo drugega ulomka.
                $$ \mathbf{\frac{a}{b}:\frac{c}{d}=\frac{a}{b}\cdot\frac{d}{c}=\frac{a\cdot d}{b\cdot c}} $$
            \end{alertblock}       

        \end{frame}



    \subsection{Številski izrazi}

        \begin{frame}[t]
            \frametitle{Številski izrazi}

            \begin{alertblock}{Vrstni red operacij}
                Pri številskih izrazih z  oklepaji \textbf{izračunamo najprej računske operacije v oklepaju}. Vedno najprej v najbolj notranjem oklepaju.

                Pri številskih izrazih brez oklepajev upoštevamo običajni vrstni red, po katerem \textbf{množimo in delimo pred seštevanjem in odštevanjem}.
            \end{alertblock}
        \end{frame}

    \subsection{Naloge z besedilo}

        \begin{frame}[t]
            \frametitle{Naloge z besedilom}

            \begin{exampleblock}{DOGOVOR}
                Vsaka naloga z besedilom zahteva tudi zapisan odgovor.
            \end{exampleblock}

            \begin{block}{POMNI}
                \begin{tabular}{|l|l|}
                    \hline 
                    operacija & rezultat \\
                    \hline \hline
                    $+$ seštevanje & vsota \\
                    \hline
                    $-$ odštevanje & razlika \\
                    \hline
                    $\cdot$ množenje & produkt \\
                    \hline
                    $:$ deljenje & kvocient \\
                    \hline 
                    
                \end{tabular}
            \end{block}
        \end{frame}

    \subsection{Izrazi s spremenljivkami}
        
        \begin{frame}[t]
            \frametitle{Izrazi s spremenljivkami}
        \end{frame}

    \subsection{Enačbe in neenačbe}
        
        \begin{frame}[t]
            \frametitle{Enačbe in neenačbe}

            \begin{alertblock}{Reševanje enačb in neenačb}
                Besedilne naloge, ki vsebujejo neznane količine (enačbe ali neenačbe) rešujemo tako, da najprej \textbf{določimo neznanko}, nato \textbf{sklepamo}, nakar \textbf{rešimo nalogo} s preglednico, diagramom ali enačbo, na koncu \textbf{preverimo rezultat} in \textbf{zapišemo odgovor}.
            \end{alertblock}
        \end{frame}

        \begin{frame}[t]
            \frametitle{Neenačbe}

            \begin{alertblock}{}
                \textbf{Neenačba} je izjavna oblika z neenakostjo ($<$, $>$, $\leq$, $\geq$) in neznanko.
            \end{alertblock}

            \begin{table}
                \centering
                \addtolength{\tabcolsep}{6pt}
                \renewcommand{\arraystretch}{1.5}                
                \begin{tabular}{||c|c||} 
                    \hhline{|t:==:t|}
                            $\mathbf{<}$ & manjše   \\ 
                    \hline
                            $\mathbf{>}$ & večje   \\ 
                    \hline
                            $\mathbf{\leq}$ & manjše ali enako   \\ 
                    \hline
                            $\mathbf{\geq}$ & večje ali enako  \\  
                    \hhline{|b:==:b|}
                \end{tabular}
            \end{table}

        \end{frame}

        \begin{frame}

            \begin{alertblock}{}
                \textbf{Osnovni množica} $\mathbf{\mathcal{U}}$ je množica števila, ki jih smemo uporabiti pri reševanju neenačbe ali neenačbe. Če osnovna množica ni posebej izbrana, je $\mathcal{U}=\mathbb{N}$. 
            \end{alertblock}

            \begin{alertblock}{}
                \textbf{Rešitev neenačbe} je vsako število, za katero dobimo iz izjavne oblike pravilno izjavo. Zapišemo množico rešitev, ki jo označimo z $\mathbf{\mathcal{R}}$. 
            \end{alertblock}

            \begin{block}{}
                Množica rešitev je odvisna od osnovne množice.
                Kadar v osnovni množici ni števila, ki reši enačbo ali neenačbo, je množica rešitev prazna. Kar zapišemo $\mathcal{R}=\emptyset$ ali $\mathcal{R}=\left\{\right\}$.
            \end{block}
        \end{frame}

        \subsection{Špela se preizkusi}
        
        \begin{frame}[t]
            \frametitle{Špela se preizkusi}
        \end{frame}